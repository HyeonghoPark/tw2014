
\section{Methodology}

Experiment starts here \cite{How the tools used}
Galvanometer is used for read intensity of the current of circuit.
Ice bath is used for maintain the low temperature.
Steam Jacket is used to make potential difference which is depend on the integral value of the  gradient of temperature.
Bitsmuth wire is used to make the electric flow, current, fluently move without loss of the intensity of the current due to the obstacle of the material of wire.

\subsection{Table}

\begin{table}[H]

  \centering

  \small

  \caption

  {

    Experimental Variables.

    $\Delta T$ denotes temperature difference between steam jacket and ice bath(${}^{\circ}C$) and $l$ denotes length of the wire($m$).

    Also, cross-sectional area of the wire(${mm}^{2}$) is represented as $A$.

    ${}^{\dag}$ These temperature differences were interpreted as relative potential differences of 1, 2, 3, 4, 5, 6, 7, 8, 9, 10, respectively.

 }

  \label{variables}

  \begin{tabular}{p{2.5cm}p{2cm}p{2cm}p{2cm}}

    \hline

    \textbf{Experiment} & \textbf{Independent Variable} & \textbf{Levels} & \textbf{Control\newline Variables}\\

    \hline

    Potential Difference & $\Delta T$ & 10, 20, 30, 40, 50, 60, 70, 80, 90, 100${}^{\dag}$ & $l = 5 m$ \newline $A = 1 {mm}^{2}$\\

    Length of Wire & $l$ & 0, 1, 2, 3, 4, 5, 6, 7, 8, 9 & $\Delta T = 50 {}^{\circ}C$ \newline $A = 1 {mm}^{2}$\\ 

    Cross-sectional Area of Wire & $A$ & 1, 1.5, 2 & $\Delta T = 50 {}^{\circ}C$ \newline $l = 5 m$\\

    \hline

  \end{tabular}

\end{table}
